\documentclass{beamer}
\special{landscape}

%\usetheme{Berlin}
\usetheme{Warsaw}

%\usecolortheme{seahorse}
%\usefonttheme[onlysmall]{structurebold}

\setbeamertemplate{headline}[split]
\setbeamertemplate{footline}[default]
\setbeamertemplate{footline}[miniframes theme]
%\logo{\includegraphics[scale=0.25]{lifia_logo.png}}

\mode<presentation>
\usepackage[spanish]{babel}
\usepackage{beamerthemesplit}
\usepackage[utf8]{inputenc}
\usepackage{color}      % use if color is used in text


% Comandos en modo Verbatim
%\usepackage{fancyvrb}


\title{Practica 3 - Parte 2 - HAND-SHAKE}
%\author{Juan Antonio Zubimendi\\azubimendi@lifia.info.unlp.edu.ar}

%\institute{LINSE}
%\date{24/04/2008}

\AtBeginSection[]

\begin{document}

\begin{frame}
%\frametitle{Presentación}
\titlepage
\end{frame}

\section{HAND-SHAKE}
\subsection{Introducción}
\begin{frame}
\frametitle{Hand-Shake}
\begin{itemize}
 \item Interfaz que nos permite comunicarnos facilmente con la Impresora, ya que realiza la temporización automáticamente
 \item Posee dos registros, de 8 bits.
  \begin{itemize}
   \item DATO: Registro de datos. De lectura y escritura. Es el caracter a enviar o el ultimo enviado.
   \item EST: Un registro de estado.
  \end{itemize}
 \item Los dos registros estan a partir de la posición 40h. 
  \begin{itemize}
      \item 40h = DATO
      \item 41h = EST
\end{itemize}
\end{itemize}
\end{frame}

\subsection{Continuación}
\begin{frame}
\frametitle{Continuación}
\begin{itemize}
 \item El registro de estado 
 \begin{itemize}
   \item Bit 0: Linea Busy - Idem Impresora
   \item Bit 1: Linea Strobe - Idem Impresora
   \item Bit 2..6: No tienen sentido
   \item Bit 7: Interrupción: 0 = Desactivada, 1 = Activada
  \end{itemize}
 \item ¿Cuándo se dispara la interrupción? Cuando la línea BUSY se desactiva.
 \item Tenemos 2 maneras de utilizar el HAND-SHAKE: con interrupciones o sin interrupciones
\end{itemize}
\end{frame}

\begin{frame}
\frametitle{Descripción}
\begin{itemize}
 \item Para utilizar HAND-SHAKE debemos usar el modo de configuración 2 (c2)
 \item La interrupción que genera el HAND-SHAKE se conectará a la interrupción de nivel 2 (INT2) del PIC
 \item Cuando el manejador de la interrupción sea invocado. La impresora va a estar lista para recibir un caracter.
 \item Podemos usar también el HAND-SHAKE por consulta de estado, sin usar interrupciónes. Es similar a usarla con el PIO.
\end{itemize}
\end{frame}

\subsection{Sin interrupciones}
\begin{frame}
\frametitle{Como usar el HAND-SHAKE sin interrpciones}
\begin{itemize}
 \item Configurar el HAND-SHAKE para que no utilice interrupciones
 \item Debemos esperar a que la impresora este lista, consultado el estado de la linea BUSY en el registro EST
 \item Cuando la impresora este lista, escribimos el caracter a imprimir en el registro DATO.
\end{itemize}
\end{frame}

\begin{frame}[fragile]
\frametitle{Configurar sin interrupciones}
\begin{itemize}
 \item Para configurar el HAND-SHAKE para que no utilice interrupciones, debemos poner a 0 el bit 7 de EST
\end{itemize}
\begin{block}{}
 \begin{verbatim}
         IN AL, EST
         AND AL, 7Fh
         OUT EST, AL
 \end{verbatim}
\end{block}
\end{frame}

\begin{frame}[fragile]
\frametitle{Enviar el caracter}
\begin{itemize}
 \item Esperar a que la impresora este lista, consultado el estado de la linea BUSY en el registro EST
 \item Cuando la impresora este lista, escribimos el caracter a imprimir en el registro DATO.
 \item ¿Qué diferencias hay con el uso de la impresora con el PIO?
\end{itemize}
\begin{block}{}
 \begin{verbatim}
   POLL: IN AL, EST
         AND AL, 1
         JNZ POLL

         MOV AL, PROX_CAR
         OUT DATO, AL
 \end{verbatim}
\end{block}
\end{frame}


\begin{frame}
\frametitle{En la práctica}
\begin{itemize}
 \item Veamos el ejercicio 3a. 
\end{itemize}
\end{frame}

\subsection{Con interrupciones}
\begin{frame}[fragile]
\frametitle{Como usar el HAND-SHAKE con interrpciones}
\begin{itemize}
 \item Configurar el HAND-SHAKE para que utilice interrupciones
 \item Debemos programar el PIC para que habilite la interrupción de nivel 2 (INT2)
 \item Debemos cargar la dirección de la rutina en el vector de interrupciones correspondiente
 \item En la rutina de la interrupción escribimos el caracter a enviar a la impresora
\end{itemize}
\end{frame}

\begin{frame}[fragile]
\frametitle{Configurar PIC}
\begin{itemize}
 \item Habilitar la INT 2 del PIC en el IMR
 \item Cargar el vetor correspondiente a la interrupción en el registro INT2
\end{itemize}
\begin{block}{}
 \begin{verbatim}
         CLI
         ...
         MOV AL, 0FBh ;  1111 1011 
         OUT IMR, AL
         MOV AL, 10
         OUT INT2, AL
         ...
 \end{verbatim}
\end{block}
\end{frame}

\begin{frame}[fragile]
\frametitle{Configurar con interrupciones}
\begin{itemize}
 \item Para configurar el HAND-SHAKE para que utilice interrupciones, debemos poner a 1 el bit 7 de EST
\end{itemize}
\begin{block}{}
 \begin{verbatim}
         IN AL, EST 
         OR AL, 80h
         OUT EST, AL
 \end{verbatim}
\end{block}
\end{frame}

\begin{frame}[fragile]
\frametitle{Como usar el HAND-SHAKE con interrpciones}
\begin{itemize}
 \item En la rutina de la interrupción escribimos el caracter a enviar a la impresora
\end{itemize}
\begin{block}{}
 \begin{verbatim}
  RUT_HAND:
         ...
         MOV AL, PROX_CAR
         OUT DATO, AL
         ...
         IRET
 \end{verbatim}
\end{block}
\end{frame}

\begin{frame}
\frametitle{En la práctica}
\begin{itemize}
 \item Veamos el ejercicio 3c. 
\end{itemize}
\end{frame}


\end{document}

