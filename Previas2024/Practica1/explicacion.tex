\documentclass{beamer}
\special{landscape}

%\usetheme{Boadilla}
%\usetheme{Hannover}
%\usetheme{Montpellier}
%\usetheme{PaloAlto}
%\usetheme{Szeged}
\usetheme{Warsaw}


\setbeamertemplate{headline}[split]
\setbeamertemplate{footline}[default]
\setbeamertemplate{footline}[miniframes theme]

\mode<presentation>

\usepackage[spanish]{babel}
\usepackage[utf8]{inputenc}
\usepackage{color}      % use if color is used in text

% Comandos en modo Verbatim
%\usepackage{fancyvrb}


\title{Practica 1 - Subrutinas}
%\author{Juan Antonio Zubimendi\\azubimendi@lifia.info.unlp.edu.ar}

\AtBeginSection[]

\begin{document}

\begin{frame}
%\frametitle{Presentación}
\titlepage
\end{frame}

\section{Ejercicio 7}

\subsection{Enunciado}
\begin{frame}
\frametitle{Ejercicio 7 - Multiplicación de números sin signo}
\begin{itemize}
\item El programa utiliza una subrutina para multiplicar dos números sin signo mayores que cero.
\item Se pasan los números a multiplicar por valor.  
\item Se para por referencia la dirección donde se debe guardar el resultado.
\item Todos los parámetros se pasan a través de la pila
\end{itemize}
\end{frame}

\subsection{Desarrollo}
\begin{frame}
\frametitle{Ejercicio 7 - Pasaje de parámetros por pila}
\includegraphics[scale=0.65]{ejercicio07/001.png}
\end{frame}

\begin{frame}
\frametitle{Ejercicio 7 - Pasaje de parámetros por pila}
\includegraphics[scale=0.65]{ejercicio07/002.png}
\end{frame}

\begin{frame}
\frametitle{Ejercicio 7 - Pasaje de parámetros por pila}
\includegraphics[scale=0.65]{ejercicio07/003.png}
\end{frame}

\begin{frame}
\frametitle{Ejercicio 7 - Pasaje de parámetros por pila}
\includegraphics[scale=0.65]{ejercicio07/004.png}
\end{frame}

\begin{frame}
\frametitle{Ejercicio 7 - Pasaje de parámetros por pila}
\includegraphics[scale=0.65]{ejercicio07/005.png}
\end{frame}

\begin{frame}
\frametitle{Ejercicio 7 - Pasaje de parámetros por pila}
\includegraphics[scale=0.65]{ejercicio07/006.png}
\end{frame}
\begin{frame}
\frametitle{Ejercicio 7 - Pasaje de parámetros por pila}
\includegraphics[scale=0.65]{ejercicio07/007.png}
\end{frame}
\begin{frame}
\frametitle{Ejercicio 7 - Pasaje de parámetros por pila}
\includegraphics[scale=0.65]{ejercicio07/008.png}
\end{frame}
\begin{frame}
  \frametitle{Ejercicio 7 - Pasaje de parámetros por pila}
\includegraphics[scale=0.65]{ejercicio07/009.png}
\end{frame}
\begin{frame}
\frametitle{Ejercicio 7 - Pasaje de parámetros por pila}
\includegraphics[scale=0.65]{ejercicio07/010.png}
\end{frame}

\begin{frame}
\frametitle{Ejercicio 7 - Pasaje de parámetros por pila}
\includegraphics[scale=0.65]{ejercicio07/011.png}
\end{frame}

\begin{frame}
\frametitle{Ejercicio 7 - Pasaje de parámetros por pila}
\includegraphics[scale=0.65]{ejercicio07/012.png}
\end{frame}

\begin{frame}
\frametitle{Ejercicio 7 - Pasaje de parámetros por pila}
\includegraphics[scale=0.65]{ejercicio07/013.png}
\end{frame}

\begin{frame}
\frametitle{Ejercicio 7 - Pasaje de parámetros por pila}
\includegraphics[scale=0.65]{ejercicio07/014.png}
\end{frame}

\begin{frame}
\frametitle{Ejercicio 7 - Pasaje de parámetros por pila}
\includegraphics[scale=0.65]{ejercicio07/015.png}
\end{frame}

\begin{frame}
\frametitle{Ejercicio 7 - Pasaje de parámetros por pila}
\includegraphics[scale=0.65]{ejercicio07/016.png}
\end{frame}

\begin{frame}
\frametitle{Ejercicio 7 - Pasaje de parámetros por pila}
\includegraphics[scale=0.65]{ejercicio07/017.png}
\end{frame}

\begin{frame}
\frametitle{Ejercicio 7 - Pasaje de parámetros por pila}
\includegraphics[scale=0.65]{ejercicio07/018.png}
\end{frame}

\begin{frame}
\frametitle{Ejercicio 7 - Pasaje de parámetros por pila}
\includegraphics[scale=0.65]{ejercicio07/019.png}
\end{frame}

\begin{frame}
\frametitle{Ejercicio 7 - Pasaje de parámetros por pila}
\includegraphics[scale=0.65]{ejercicio07/020.png}
\end{frame}

\begin{frame}
\frametitle{Ejercicio 7 - Pasaje de parámetros por pila}
\includegraphics[scale=0.65]{ejercicio07/021.png}
\end{frame}

\begin{frame}
\frametitle{Ejercicio 7 - Pasaje de parámetros por pila}
\includegraphics[scale=0.65]{ejercicio07/022.png}
\end{frame}

\section{Ejercicio 10}

\subsection{Enunciado}

\begin{frame}
\frametitle{Ejercicio 10 - SWAP (intercambio) }
\begin{itemize}
\item Escribir una subrutina SWAP que intercambie dos datos de 16 bits almacenados en memoria.
\item Los parámetros deben ser pasados por referencia desde el programa principal a través de la pila.  
\end{itemize}
\end{frame}

\subsection{Desarrollo}

\begin{frame}
\frametitle{Ejercicio 10 - SWAP}
\includegraphics[scale=0.65]{ejercicio10/000.png}
\end{frame}

\begin{frame}
\frametitle{Ejercicio 10 - SWAP}
\includegraphics[scale=0.65]{ejercicio10/001.png}
\end{frame}

\begin{frame}
\frametitle{Ejercicio 10 - SWAP}
\includegraphics[scale=0.65]{ejercicio10/002.png}
\end{frame}

\begin{frame}
\frametitle{Ejercicio 10 - SWAP}
\includegraphics[scale=0.65]{ejercicio10/003.png}
\end{frame}

\begin{frame}
\frametitle{Ejercicio 10 - SWAP}
\includegraphics[scale=0.65]{ejercicio10/004.png}
\end{frame}

\begin{frame}
\frametitle{Ejercicio 10 - SWAP}
\includegraphics[scale=0.65]{ejercicio10/005.png}
\end{frame}

\begin{frame}
\frametitle{Ejercicio 10 - SWAP}
\includegraphics[scale=0.65]{ejercicio10/006.png}
\end{frame}

\begin{frame}
\frametitle{Ejercicio 10 - SWAP}
\includegraphics[scale=0.65]{ejercicio10/007.png}
\end{frame}

\begin{frame}
\frametitle{Ejercicio 10 - SWAP}
\includegraphics[scale=0.65]{ejercicio10/008.png}
\end{frame}

\begin{frame}
\frametitle{Ejercicio 10 - SWAP}
\includegraphics[scale=0.65]{ejercicio10/009.png}
\end{frame}

\begin{frame}
\frametitle{Ejercicio 10 - SWAP}
\includegraphics[scale=0.65]{ejercicio10/010.png}
\end{frame}

\begin{frame}
\frametitle{Ejercicio 10 - SWAP}
\includegraphics[scale=0.65]{ejercicio10/011.png}
\end{frame}

\begin{frame}
\frametitle{Ejercicio 10 - SWAP}
\includegraphics[scale=0.65]{ejercicio10/012.png}
\end{frame}

\begin{frame}
\frametitle{Ejercicio 10 - SWAP}
\includegraphics[scale=0.65]{ejercicio10/013.png}
\end{frame}

\begin{frame}
\frametitle{Ejercicio 10 - SWAP}
\includegraphics[scale=0.65]{ejercicio10/014.png}
\end{frame}

\begin{frame}
\frametitle{Ejercicio 10 - SWAP}
\includegraphics[scale=0.65]{ejercicio10/015.png}
\end{frame}

\begin{frame}
\frametitle{Ejercicio 10 - SWAP}
\includegraphics[scale=0.65]{ejercicio10/016.png}
\end{frame}

\begin{frame}
\frametitle{Ejercicio 10 - SWAP}
\includegraphics[scale=0.65]{ejercicio10/017.png}
\end{frame}

\begin{frame}
\frametitle{Ejercicio 10 - SWAP}
\includegraphics[scale=0.65]{ejercicio10/018.png}
\end{frame}

\begin{frame}
\frametitle{Ejercicio 10 - SWAP}
\includegraphics[scale=0.65]{ejercicio10/019.png}
\end{frame}

\begin{frame}
\frametitle{Ejercicio 10 - SWAP}
\includegraphics[scale=0.65]{ejercicio10/020.png}
\end{frame}

\begin{frame}
\frametitle{Ejercicio 10 - SWAP}
\includegraphics[scale=0.65]{ejercicio10/021.png}
\end{frame}

\begin{frame}
\frametitle{Ejercicio 10 - SWAP}
\includegraphics[scale=0.65]{ejercicio10/022.png}
\end{frame}

\begin{frame}
\frametitle{Ejercicio 10 - SWAP}
\includegraphics[scale=0.65]{ejercicio10/023.png}
\end{frame}

\begin{frame}
\frametitle{Ejercicio 10 - SWAP}
\includegraphics[scale=0.65]{ejercicio10/024.png}
\end{frame}

\begin{frame}
\frametitle{Ejercicio 10 - SWAP}
\includegraphics[scale=0.65]{ejercicio10/025.png}
\end{frame}

\begin{frame}
\frametitle{Ejercicio 10 - SWAP}
\includegraphics[scale=0.65]{ejercicio10/026.png}
\end{frame}

\begin{frame}
\frametitle{Ejercicio 10 - SWAP}
\includegraphics[scale=0.65]{ejercicio10/027.png}
\end{frame}

\begin{frame}
\frametitle{Ejercicio 10 - SWAP}
\includegraphics[scale=0.65]{ejercicio10/028.png}
\end{frame}

\begin{frame}
\frametitle{Ejercicio 10 - SWAP}
\includegraphics[scale=0.65]{ejercicio10/029.png}
\end{frame}

\subsection{Observaciones}

\begin{frame}
\frametitle{Ejercicio 10 - SWAP (intercambio) }
El programa cumple con el enunciado, pero...
\begin{itemize}
\item No desapila los dos parámetros de la pila
\item La subrutina preserva los registros AX, BX y DX. Pero modifica los registro BX, CX y DX. No es necesario preservar AX, CX si.
\end{itemize}
\end{frame}

\end{document}

