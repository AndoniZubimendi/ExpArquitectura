\documentclass{beamer}
\special{landscape}

%\usetheme{Berlin}
\usetheme{Warsaw}

%\usecolortheme{seahorse}
%\usefonttheme[onlysmall]{structurebold}

\setbeamertemplate{headline}[split]
\setbeamertemplate{footline}[default]
\setbeamertemplate{footline}[miniframes theme]

\mode<presentation>
\usepackage[spanish]{babel}
\usepackage{beamerthemesplit}
\usepackage{color}      % use if color is used in text

\usepackage[utf8]{inputenc}

% Comandos en modo Verbatim
%\usepackage{fancyvrb}

\usepackage{listings}


%%
%% WinMIPS64 definition (c) 2013
\lstdefinelanguage{WinMIPS64}
 {morekeywords={%
cvt.l.d,cvt.d.l,mfc1,mtc1,DADD,DADDI,ANDI,XORI,HALT,BEQ,BNEQ,LD,NOP,DMUL,DSUB,AND,DDIV,%
	JAL,J,DSLL,BNEZ,SD},%
 otherkeywords={.word,.code,.data,.text},%
   sensitive=false,%
   morecomment=[l]{;},%
   morestring=[b]",%
 }

\lstset{emph={%  
    F0,F1,F2,F3,F4,F5,F6,F7,F8,F9,F10,F11,F12,F13,F14,F15,F16,F17,F18,F19%
    F20,F21,F22,F23,F24,F25,F26,F27,F28,F29,F30,F31%
    f0,f1,f2,f3,f4,f5,f6,f7,f8,f9,f10,f11,f12,f13,f14,f15,f16,f17,f18,f19%
    f20,f21,f22,f23,f24,f25,f26,f27,f28,f29,f30,f31%
    R0,R1,R2,R3,R4,R5,R6,R7,R8,R9,R10,R11,R12,R13,R14,R15,R16,R17,R18,R19%
    R20,R21,R22,R23,R24,R25,R26,R27,R28,R29,R30,R31%
    r0,r1,r2,r3,r4,r5,r6,r7,r8,r9,r10,r11,r12,r13,r14,r15,r16,r17,r18,r19%
    r20,r21,r22,r23,r24,r25,r26,r27,r28,r29,r30,r31%
	$0,$zero,$s0,$s1,$s2,$s3,$s4,$s5,$s6,$s7,$sp,$t1%
    },emphstyle={\color{red}\bfseries}%
}%

%\lstset{emph={%  
%    .code,.data,.word
%    },emphstyle={\color{green}\bfseries}%
%}%
\title{Practica 5 - WinMIPS64}

\AtBeginSection[]

\begin{document}

\section{Generación y manejo de pila}
\subsection{Introducción}
\begin{frame}
\frametitle{Introducción}
En esta arquitectura no tenemos soporte nativo para el manejo de pilas. \\
Veamos en que consiste el manejo de pila del MSX88:
\begin{itemize}
\item Tiene un registro SP, que se usa solo para el manejo de la pila y que inicialmente vale 08000h.
\item Tenemos una instrucción PUSH que nos permite \emph{apilar} valores de 16 bits en la misma.
\item Tenemos una instrucción POP que nos permite \emph{desapilar} valores de 16 bits de la misma.
\item La pila también se utiliza para el manejo de subrutinas, pero ya tenemos un reemplazo para eso.
\end{itemize}
\end{frame}

\subsection{Inicialización}
\begin{frame}[fragile]
\frametitle{Pila}
\begin{itemize}
\item El MSX88 posee un registro especifico para la Pila
\item WinMIPS64 posee 32 registros de proposito general, podemos elegir uno y usarlo explusivamente para la pila. 
\item La convención de nombres de registros ya vistos ya contempla al registro \emph{r29} como el registro \emph{\$sp}.
\item El registro SP esta inicializado en 08000h, debemos inicializar el registro de pila a un valor adecuado.
\item Podemos inicializar el registro \emph{\$sp} en 0400h, que es la posición de memoria de datos más alta que tenemos en WinMIPS64.
\end{itemize}

\begin{block}{}
\begin{lstlisting}[language=WinMIPS64,basicstyle=\ttfamily,keywordstyle=\color{blue}]
    .code
    daddi $sp, $0, 0x400
    ...
\end{lstlisting}
\end{block}

\end{frame}

\subsection{Apilando Valores}
\begin{frame}
\frametitle{Apilar}
La instrucción PUSH del MSX88 es la encargada de apilar un valor en la pila. Veamos en que consiste apilar un valor:
\begin{itemize}
\item $(SP) \xleftarrow{} (SP) - 2$ \emph{; Decremento Puntero de pila}
\item $[SP+1:SP] \xleftarrow{} (fuente)$ \emph{; Guardo valor en la pila}
\end{itemize}
Para simular esta operación deberiamos:
\begin{itemize}
\item Decrementar en 8 el puntero de pila. ¿Por qué 8?
\item Guardar en la posición de memoria del puntero de pila el valor.
\end{itemize}

\end{frame}

\begin{frame}[fragile]
\frametitle{Apilar - Detalle}


\begin{block}{Apilando el valor del registro \$t1}
\begin{lstlisting}[language=WinMIPS64,basicstyle=\ttfamily,keywordstyle=\color{blue}]
    ...
    daddi $sp, $sp, -8
    sd $t1, 0($sp)      ; Guardo en la pila
    ...
\end{lstlisting}
\end{block}

\end{frame}



\subsection{Sacando valores de la pila}
\begin{frame}
\frametitle{Desapilar}
La instrucción POP del MSX88 es la encargada de desapilar un valor de la pila. Veamos en que consiste:
\begin{itemize}
\item $(fuente) \xleftarrow{} [SP+1:SP]$ \emph{; Guardo valor de la pila}
\item $(SP) \xleftarrow{} (SP) + 2$ \emph{; Incremento el Puntero de pila}
\end{itemize}
Para simular esta operación deberiamos:
\begin{itemize}
\item Leer el dato de la posición de memoria del puntero de pila y guardarlo en un registro.
\item Incrementar en 8 el puntero de pila.
\end{itemize}
\end{frame}

\begin{frame}[fragile]
\frametitle{Desapilar - Detalle}


\begin{block}{Desapilando en el registro \$s1}
\begin{lstlisting}[language=WinMIPS64,basicstyle=\ttfamily,keywordstyle=\color{blue}]
    ...
    ld $s1, 0($sp)      ; Leo desde la pila
    daddi $sp, $sp, +8
    ...
\end{lstlisting}
\end{block}

\end{frame}


\subsection{Multiples Valores}

\subsection{Sacando valores de la pila}
\begin{frame}[fragile]
\frametitle{Multiples valores}
Veamos que pasa si se apilan varios valores seguidos:

\tiny{
\begin{block}{}
\begin{lstlisting}[language=WinMIPS64,basicstyle=\ttfamily,keywordstyle=\color{blue}]
    ...
    daddi $sp, $sp, -8
    sd $s1, 0($sp)      ; Apilo s1
	
    daddi $sp, $sp, -8
    sd $s2, 0($sp)      ; Apilo s2
	
    daddi $sp, $sp, -8
    sd $s3, 0($sp)      ; Apilo s3
	
    daddi $sp, $sp, -8
    sd $s4, 0($sp)      ; Apilo s4

    ...
\end{lstlisting}
\end{block}
}

Podriamos agrupar los \emph{daddi} y ver si podemos tener menos instrucciones.

\end{frame}


\begin{frame}[fragile]
\frametitle{Multiples valores}
\tiny{
\begin{block}{}
\begin{lstlisting}[language=WinMIPS64,basicstyle=\ttfamily,keywordstyle=\color{blue}]
    ...
    daddi $sp, $sp, -32 ; Reservo 32 bytes
    sd $s1, 24($sp) ; Apilo s1
    sd $s2, 16($sp) ; Apilo s2
    sd $s3, 8($sp)  ; Apilo s3
    sd $s4, 0($sp)  ; Apilo s4
    ...
\end{lstlisting}
\end{block}
}
\begin{itemize}
\item Ajustamos \emph{\$sp} al inicio, le restamos 8 por cada registro a apilar.
\item Luego guardamos cada registro, desplazandolo de a 8 posiciones.
\end{itemize}

\end{frame}


\begin{frame}[fragile]
\frametitle{Multiples valores - Continuación}
Y para desapilar, lo hacemos de manera similar
\tiny{
\begin{block}{}
\begin{lstlisting}[language=WinMIPS64,basicstyle=\ttfamily,keywordstyle=\color{blue}]
    ...

    ld $s1, 24($sp) ; Restauro s1
    ld $s2, 16($sp) ; Restauro s2
    ld $s3, 8($sp)  ; Restauro s3
    ld $s4, 0($sp)  ; Restauro s4
    daddi $sp, $sp, +32
    ...
\end{lstlisting}
\end{block}
}
\end{frame}


\subsection{Prólogo y Epílogo de Subrutinas}

\begin{frame}
\frametitle{Pilas y Subrutina}
Toda subrutina se dividirá siempre en tres partes
\begin{itemize}
\item \emph{Prólogo}: se resguarda en la pila todos los registros que deban ser preservados
\item \emph{Cuerpo} de la subrutina: con el código propio de la misma
\item \emph{Epílogo}: se restauran los registros preservados en el prólogo
\end{itemize}
\end{frame}

\begin{frame}[fragile]
\frametitle{}
\tiny{
\begin{block}{}
\begin{lstlisting}[language=WinMIPS64,basicstyle=\ttfamily,keywordstyle=\color{blue}]

 subrutina:
    daddi $sp, $sp, -16 
    sd $ra, 8($sp) ; Apilo ra
    sd $s0, 0($sp)  ; Apilo s0
	
    ...

    ld $ra, 8($sp)  ; Restauro ra
    ld $s0, 0($sp)  ; Restauro s0
    daddi $sp, $sp, +16
    jr $ra  ; $ra tiene valor de retorno
\end{lstlisting}
\end{block}
}
\begin{itemize}
\item Si no se modifican los registros \emph{\$s0} a \emph{\$s7}, no es necesario guardarlos
\item Una subrutina sencilla podria tener un prólogo y epílogo vacio
\end{itemize}

\end{frame}

\subsection{Anidamiento de subrutinas}
\begin{frame}
\frametitle{Anidamiento de Subrutinas}

\begin{itemize}
\item Si una subrutina llama a otra subrutina, va a alterar el valor del registro \emph{\$ra}. Por lo tanto debemos preservar el valor que recibimos de \emph{\$ra}.
\item Si no llama a otra subrutina, no es necesario que guarde el valor contenido en \emph{\$ra}.
\item Esto es valido tanto para subrutinas que se llamen a si mismas (recursivas) como para subrutinas que llamen a otras.
\end{itemize}

\end{frame}


\begin{frame}[fragile]
\frametitle{Ejemplo}
\tiny{
\begin{block}{}
\begin{lstlisting}[language=WinMIPS64,basicstyle=\ttfamily,keywordstyle=\color{blue}]
.data
valor: .word 10
result: .word 0
.text
    daddi $sp, $0, 0x400 ; Inicializa $sp
    ld $a0, valor($0)
    jal factorial
    sd $v0, result($0)
    halt
factorial: daddi $sp, $sp, -16
           sd $ra, 0($sp)
           sd $s0, 8($sp)
           beqz $a0, fin_rec
           dadd $s0, $0, $a0
           daddi $a0, $a0, -1
           jal factorial
           dmul $v0, $v0, $s0
           j fin		   
fin_rec:   daddi $v0, $0, 1
fin:       ld $s0, 8($sp)
           ld $ra, 0($sp)
           daddi $sp, $sp, 16
           jr $ra		   
\end{lstlisting}
\end{block}
}

\end{frame}



\end{document}

