\documentclass{beamer}
\special{landscape}

%\usetheme{Berlin}
\usetheme{Warsaw}

%\usecolortheme{seahorse}
%\usefonttheme[onlysmall]{structurebold}

\setbeamertemplate{headline}[split]
\setbeamertemplate{footline}[default]
\setbeamertemplate{footline}[miniframes theme]
%\logo{\includegraphics[scale=0.25]{lifia_logo.png}}

\mode<presentation>
\usepackage[spanish]{babel}
\usepackage{beamerthemesplit}
\usepackage{color}      % use if color is used in text

\usepackage[utf8]{inputenc}

% Comandos en modo Verbatim
%\usepackage{fancyvrb}

\usepackage{listings}


%%
%% WinMIPS64 definition (c) 2013
\lstdefinelanguage{WinMIPS64}
 {morekeywords={%
cvt.l.d,cvt.d.l,mfc1,mtc1,DADD,DADDI,ANDI,XORI,HALT,BEQ,BNEQ,LD,NOP,DMUL,DSUB,AND,DDIV,%
	beqz,jal,jr,slt,JAL,J,DSLL,BNEZ,SD},%
 otherkeywords={.word,.code,.data,.text},%
   sensitive=false,%
   morecomment=[l]{;},%
   morestring=[b]",%
 }

\lstset{emph={%  
    F0,F1,F2,F3,F4,F5,F6,F7,F8,F9,F10,F11,F12,F13,F14,F15,F16,F17,F18,F19%
    F20,F21,F22,F23,F24,F25,F26,F27,F28,F29,F30,F31%
    f0,f1,f2,f3,f4,f5,f6,f7,f8,f9,f10,f11,f12,f13,f14,f15,f16,f17,f18,f19%
    f20,f21,f22,f23,f24,f25,f26,f27,f28,f29,f30,f31%
    R0,R1,R2,R3,R4,R5,R6,R7,R8,R9,R10,R11,R12,R13,R14,R15,R16,R17,R18,R19%
    R20,R21,R22,R23,R24,R25,R26,R27,R28,R29,R30,R31%
    r0,r1,r2,r3,r4,r5,r6,r7,r8,r9,r10,r11,r12,r13,r14,r15,r16,r17,r18,r19%
    r20,r21,r22,r23,r24,r25,r26,r27,r28,r29,r30,r31%
    },emphstyle={\color{red}\bfseries}%
}%

%\lstset{emph={%  
%    .code,.data,.word
%    },emphstyle={\color{green}\bfseries}%
%}%
\title{Practica 5 - WinMIPS64}
%\author{Juan Antonio Zubimendi\\azubimendi@lifia.info.unlp.edu.ar}

\AtBeginSection[]

\begin{document}

\section{Punto Flotante}

\subsection{Introducción}
\begin{frame}
\frametitle{Introducción}
\begin{itemize}
\item Lectura de memoria: \emph{l.d f1, desp(r5)}
\item Escritura a memoria: \emph{s.d f3, res(r0)}
\item Copiar valor de un registro a otro: \emph{mov.d fd, ff ; fd = ff }
\end{itemize}
No hay carga de valores inmediatos!
\end{frame}


\begin{frame}
\frametitle{Operaciones básicas}
\begin{itemize}
\item \emph{add.d fd, ff, fg}: Suma ff con fg, dejando el resultado en fd
\item \emph{sub.d fd, ff, fg}: Resta fg a ff, dejando el resultado en fd
\item \emph{mul.d fd, ff, fg}: Multiplica ff con fg, dejando el resultado en fd
\item \emph{div.d fd, ff, fg}: Divide ff por fg, dejando el resultado en fd
\end{itemize}
\end{frame}

\subsection{Conversión Punto Fijo / Punto Flotante}

\begin{frame}
\frametitle{Conversión Numero Entero / Punto Flotante}
Las copias se deben hacer en 2 pasos.
\begin{itemize}
\item Copia de bits desde/hasta punto flotante
\begin{itemize}
\item En este paso se copian bits, no se copian valores!
\end{itemize}
\item Conversión de los bits en número entero / punto flotante.
\end{itemize}

Las conversiones no siempre respetan el número original (por cuestiones de redondeo y representación). Nosotros no vamos a tener en cuenta esto en la práctica.
\end{frame}

\begin{frame}[fragile]
\frametitle{De registro entero a punto flotante}
Para copiar el valor que tengo en un registro entero (r0 a r31) a uno de punto flotante (f0 a f31):
\begin{itemize}
\item Copiar los 64 bits del registro entero rf al registro fd de punto flotante
\begin{itemize}
\item \emph{mtc1 rf, fd}
\end{itemize}

\item Conviertir a punto flotante el valor entero copiado al registro ff, dejándolo en fd
\begin{itemize}
\item \emph{cvt.d.l fd, ff}
\end{itemize}

\end{itemize}

Importante: los números muy grandes serán redondeados en su mejor representación de punto flotante.

\begin{block}{Copiar el valor de r5 a f10}
\begin{lstlisting}[language=WinMIPS64,basicstyle=\ttfamily,keywordstyle=\color{blue}]
mtc1 r5, f1
cvt.d.l f10, f1
\end{lstlisting}
\end{block}

\end{frame}

\begin{frame}[fragile]
\frametitle{De punto flotante a registro entero}
Para copiar el valor que tengo en un registro de punto flotante (f0 a f31) a un registro entero (r1 a r31):
\begin{itemize}
\item Conviertir a entero el valor en punto flotante contenido en ff, dejándolo en fd
\begin{itemize}
\item \emph{cvt.l.d fd, ff}
\end{itemize}
\item Copiar los 64 bits del registro ff de punto flotante al registro rd entero
\begin{itemize}
\item \emph{mfc1 rd, ff}
\end{itemize}
\end{itemize}

Importante: El número se trunca, no se redondea
\begin{block}{Copiar el valor de f3 a r8}
\begin{lstlisting}[language=WinMIPS64,basicstyle=\ttfamily,keywordstyle=\color{blue}]
cvt.l.d f4, f3
mfc1 r8, f4
\end{lstlisting}
\end{block}

\end{frame}


\subsection{Saltos condicionales en Punto Flotante}

\begin{frame}
\frametitle{Saltos condicionales en Punto Flotante}
La unidad de punto flotante posee un flag (FP):
\begin{itemize}
\item \emph{c.lt.d fd, ff}: Compara fd con ff, dejando FP=1 si fd es menor que ff, sino FP=0
\item \emph{c.le.d fd, ff}: Compara fd con ff, dejando flag FP=1 si fd es menor o igual que ff, sino FP=0
\item \emph{c.eq.d fd, ff}: Compara fd con ff, dejando flag FP=1 si fd es igual que ff, sino FP=0
\end{itemize}

Podemos usar el flag FP con las siguientes instrucciones:
\begin{itemize}
\item \emph{bc1t etiqueta}: Salta a \emph{etiqueta} si flag FP=1 
\item \emph{bc1f etiqueta}: Salta a \emph{etiqueta} si flag FP=0
\end{itemize}
\end{frame}


\section{Subrutinas}
\subsection{Subrutinas}
\begin{frame}
\frametitle{Subrutinas}
\begin{itemize}
\item El soporte de la arquitectura para la invocación de subrutinas es menor que el soporte que existe en otras arquitecturas (Pila, CALL, RET)
\item La instrucción encargada de realizar un llamado a una subrutina es \emph{jal} (jump and link):
\begin{itemize}
\item La instrucción guarda en el registro \emph{r31} la dirección de retorno de la subrutina
\item Cambia el PC por la dirección de la subrutina
\end{itemize}
\item Para retornar, basta retornar a la posición apuntada por el registro \emph{r31}
\begin{itemize}
\item Esto podemos hacerlo con la instrucción \emph{jr r31}
\end{itemize}
\end{itemize}
\end{frame}



\begin{frame}
\frametitle{Subrutinas}

\tiny{
\begin{tabular}{| c | p{3.5cm} | p{3.5cm} |}
\hline
Acción & MSX88 & WinMIPS64 \\ \hline
Llamada a subrutina &
\parbox{3.5cm}{CALL \emph{subrutina} \\ 
	Se apila la dirección de retorno \\ 
	Se actualiza IP con el valor de la subrutina } & 
\parbox{3.5cm}{jal \emph{subrutina} \\
Se guarda en el registro \emph{r31} la dirección de retorno \\
Se actualiza PC con el valor de la subrutina} 
\\ & & \\ \hline


Retorno de la subrutina &
\parbox{3.5cm}{RET \\ 
	Se desapila la dirección de retorno y se  actualiza IP con ese valor } & 
\parbox{3.5cm}{jr \emph{r31} \\
Se actualiza PC con el valor del registro \emph{r31}, que tiene la dirección de retorno de la subrutina} 
\\ & & \\ \hline
		
\end{tabular}	
}

\begin{itemize}
\item ¿Qué pasa si una subrutina invoca a otra subrutina?
\end{itemize}


\end{frame}


\begin{frame}[fragile]
\frametitle{Ejemplo}
\tiny{
\begin{block}{}
\begin{lstlisting}[language=WinMIPS64,basicstyle=\ttfamily,keywordstyle=\color{blue}]
               .data
               base: .word 16
               exponente: .word 4
               result: .word 0
			   
               .text
               ld r1, base(r0)
               ld r2, exponente(r0)
               jal a_la_potencia
               sd r5, result(r0)
               halt

a_la_potencia: daddi r5, r0, 1
lazo:          beqz r2, terminar
               daddi r2, r2, -1
               dmul r5, r5, r1
               j lazo
terminar:      jr r31
\end{lstlisting}
\end{block}
}
\end{frame}



\section{Uso normalizado de los registros}
\begin{frame}
\frametitle{Normalización de registros}
En lugar de usar r0-r31 es más conveniente darle nombre más significativos a los registros:
\tiny{
\begin{tabular}{| c | c | p{6.5cm} | c |}
\hline
Registro & Nombre & Uso & Preservado \\ \hline
\emph{r0} & \emph{\$zero}, {\$0} & Siempre tiene el valor 0 y no se puede cambiar. & No \\ \hline
\emph{r1} & \emph{\$at} & Assembler Temporary – Reservado para ser usado por el ensamblador. & No \\ \hline
\emph{r2 - r3} & \emph{\$v0 - \$v1} & Valores de retorno de la subrutina llamada. & No \\ \hline
\emph{r4 - r7} & \emph{\$a0 - \$a3} & Argumentos pasados a la subrutina llamada. & No \\ \hline
\emph{r8 - r15} & \emph{\$t0 - \$t7} & Registros temporarios. No son conservados en el llamado a subrutinas. & No \\ \hline
\emph{r16 - r23} & \emph{\$s0 - \$s7} &  Registros salvados durante el llamado a subrutinas. & Si \\ \hline
\emph{r24 - r25} & \emph{\$t8 - \$t9} & Registros temporarios. No son conservados en el llamado a subrutinas. & No \\ \hline
\emph{r26 - r27} & \emph{\$k0 - \$k1} & Para uso del kernel del sistema operativo. & No \\ \hline
\emph{r28} & \emph{\$gp} & Global Pointer – Puntero a la zona de la memoria estática del programa. & Si \\ \hline
\emph{r29} & \emph{\$sp} & Stack Pointer – Puntero al tope de la pila. & Si \\ \hline
\emph{r30} & \emph{\$fp} & Frame Pointer – Puntero al marco actual de la pila. & Si \\ \hline
\emph{r31} & \emph{\$ra} & Return Address – Dirección de retorno en un llamado a una subrutina. & Si \\ \hline
		
\end{tabular}	
}
\begin{itemize}
\item Preservado implica que los registros deben devolverse \emph{inalterados} desde una subrutina.
\item \emph{\$s0-\$s7} representan las variables locales de la subrutina / programa.
\item \emph{\$t0-\$t9} son registros para almacenar resultados auxiliares.
\item Este uso de registros es el recomendado y esta normalizado
\item Si una subrutina altera el valor de algún registro que debe ser preservado, debe conservar el valor original para poder restaurarlo.
\end{itemize}

\end{frame}


\begin{frame}[fragile]
\frametitle{Ejemplo}
\tiny{
\begin{block}{}
\begin{lstlisting}[language=WinMIPS64,basicstyle=\ttfamily,keywordstyle=\color{blue}]
               .data
               base: .word 16
               exponente: .word 4
               result: .word 0
			   
               .text
               ld $a0, base($0)
               ld $a1, exponente($0)
               jal a_la_potencia
               sd $v0, result($0)
               halt

a_la_potencia: daddi $v0, $0, 1
lazo:          beqz $a1, terminar
               daddi $a1, $a1, -1
               dmul $v0, $v0, $a0
               j lazo
terminar:      jr $ra
\end{lstlisting}
\end{block}
}
\end{frame}


\end{document}

